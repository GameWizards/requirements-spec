\documentclass[12pt]{article}
\usepackage{graphicx}
\usepackage{hyperref}
\usepackage{cite}
\usepackage{mdframed}

\begin{document}
\title{CSE326 Semester Project Requirement Spec: Anttris}
\author{Chris Aikman\\Benji Cope\\Skyler Manzanares\\Hugo Rivera\\Sean Turner}
\maketitle
% abstract
\begin{abstract}
Lorem Ipsum
\end{abstract}
% overview
\section{Project Overview}
Anttris is designed to be a game that offers both creative freedom and a competitive edge. The game revolves around solving puzzle cubes which are composed of different types of blocks. Each block has a special property allowing for sophisticated puzzles to be created. Players can solve the blocks by interacting with these blocks, doing things from removing blocks from the cube to setting off massive chains!

Anttris will include two different game-mode categories: competitive game modes, and single-player game modes. Single-player game modes will focus on clearing puzzles with emphasis placed on efficiency of the solution or solution time. Competitive game modes shift the focus to solving cubes faster than an opponent.

One central game concept to Anttris is the ability to creat your own puzzle blocks. Players will be able to create custom blocks that they can use when playing competitively. The goal of the game here is no longer to simply solve a cube faster than your opponent; you now also want to \textsl{create} a puzzle that will confuse your opponent long enough to solve theirs first.
\subsection{Scope and Objectives}
\subsection{Supplementary Requirements}
% customer reqs
\subsubsection{Interface Requirements}
To make our video game easy to use, it is required that the user interface
provide intuitive interaction. To be able to make our game accessible to
the standard user, we have adopted as a requirement that standard
input devices -- namely a mouse -- be supported.

Graphical User Interfaces are to be simplistic, and not provide an overwhelming
volume of functionallity. Menu systems should contain no more than five (5)
functional options. Menus should follow a logical tree and \textsl{aid} in
game navigation.
\subsubsection{Performance Requirements}
As Anttris is to be professional-grade software, it is both necessary and
sufficient that it run quickly. As is the standard for video-games, Anttris
will deliver a minimum of 30 frames per second, and a max of 60 frames per
second. Any speed outside of this range on a modern-day computer is hereby
defined unacceptable.

%How do we define a modern day computer? regular old office computer?
As previously mentioned, Anttris is designed to be a game available to a common
computer user. To this end, the game is required to meet the above-defined
speed standard when tested on computers commonly used in a traditional office
workspace whose primary purpose is word-processing and web browsing.
\section{Customer Requirements}
% reqs analysis
\section{Requirements Analysis}

\subsection{Scope and Objectives}

\subsection{Supplementary Requirements}

% customer reqs
\section{Customer Requirements}
\subsection{User-Case Diagrams}
\subsection{Actor Descriptions}
\subsection{Use-case Descriptions}
\begin{mdframed}
    \subsubsection{Host match}
    \begin{description}
        \item[Entry conditions] Internet connection
        \item[Exit conditions] None. Will go back to game menu.
        \item[Participating Actors] Online player and local player
        \item[Flow of events]:
            \begin{enumerate}
                \item User creates a game, selects number of players and game
                    type
                \item User shares (friendly looking) match ID with other
                    players (I’m not sure about this…)
                \item These players may connect to the match
                \item The game ends when all players disconnect.
            \end{enumerate}
    \end{description}
\end{mdframed}

% reqs analysis
\section{Requirements Analysis}

\subsection{Structural Analysis}

\subsection{Behavioral Analysis}

\subsection{User Interface Analysis}

% validation and criteria
\section{Validation and Criteria}

% appendices
\section{Appendices}

\subsection{Project Status}
\end{document}







